\documentclass[a4paper,landscape]{article} % Dokumentenklasse für A4-Seiten im Querformat
\usepackage[utf8]{inputenc} % Paket zur Unterstützung von UTF-8-Zeichenkodierung
\usepackage{geometry} % Paket zur Anpassung der Seitenränder
\usepackage{tikz} % Paket zur Erstellung von Grafiken und Zeichnungen
\usepackage{array} % Paket zur Verbesserung der Tabellenformatierung
\usepackage{fancyhdr} % Paket zur Anpassung der Kopf- und Fußzeilen
\usepackage{xcolor} % Paket zur Verwendung von Farben
\usepackage{forloop} % Paket zur Verwendung von for-Schleifen

% Anpassung der Seitenränder
\geometry{left=0.5cm,right=0.5cm,top=1cm,bottom=1cm}

% Seitenstil festlegen, um die Seitennummerierung auszuschalten
\pagestyle{empty}

\begin{document}

% Definition der Kreisgröße und des Abstands
\newcommand{\circlesize}{0.4cm} % Größe der Kreise
\newcommand{\circlemargin}{1.1} % Faktor für den Abstand der Kreise (relativ zur Kreisgröße)
\newcommand{\groupmargin}{1.6} % Faktor für den Abstand zwischen Gruppen von 8 Kreisen (relativ zur Kreisgröße)
\newcounter{rownum} % Zähler für die Zeilenanzahl
\newcounter{totalrows} % Gesamtanzahl der Zeilen

% Setzen der Gesamtanzahl der Zeilen
\setcounter{totalrows}{16} % Hier die gewünschte Anzahl der Zeilen anpassen

% Definition eines neuen Befehls zur Erstellung der Getränkekreise
\newcommand{\drinkcircles}{
    \begin{tikzpicture}[baseline={(current bounding box.center)}]
        % Schleife zur Erstellung von 48 Kreisen
        \foreach \i in {1,...,48} {
            % Berechnet den zusätzlichen Abstand zwischen Gruppen von 8 Kreisen
            \pgfmathsetmacro{\extra}{int((\i-1)/8)*(\groupmargin-\circlemargin)*\circlesize}
            % Zeichnet einen Kreis an der Position (\i * Abstand + zusätzlicher Abstand, 0)
            \node at (\i*\circlemargin*\circlesize + \extra, 0) [draw, circle, minimum size=\circlesize] {};
        }
    \end{tikzpicture}
}

\centering % Zentriert die Tabelle auf der Seite
\renewcommand{\arraystretch}{0.5} % Verringerung der Zeilenhöhe
\begin{tabular}{|m{5cm}|p{22.1cm}|} % Definition der Tabellenspalten mit festen Breiten
    \hline
    \textbf{Name} & \textbf{Getränke} \\ % Kopfzeile der Tabelle
    \hline
    \forloop{rownum}{1}{\value{rownum} < \value{totalrows}}{%
        \rule{0pt}{1cm} & \drinkcircles \\ % Zeile mit Kreisen
        \hline
    }
\end{tabular}

\vspace{1cm} % Platz zwischen Tabelle und Legende

% Legende unter der Tabelle
\begin{tikzpicture}
    % Grüner Kreis für bezahltes Getränk
    \node at (0,0) [draw, circle, fill=green, minimum size=\circlesize] {};
    \node at (2,0) {Bezahltes Getränk}; % Text neben dem grünen Kreis

    % Durchgestrichener Kreis für getrunkenes Getränk
    \node at (6,0) [draw, circle, minimum size=\circlesize] {};
    \draw (5.85, -0.15) -- (6.15, 0.15); % Linie von links unten nach rechts oben
    \node at (8,0) {Getrunkenes Getränk}; % Text neben dem durchgestrichenen Kreis

    % Preisinformation
    \node at (14,0) {Ein Getränk kostet 1€};
\end{tikzpicture}

\end{document}
